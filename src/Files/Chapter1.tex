\section{Introduction}
\label{sec:introduction}
\subsection{Background}

The concept of a smart home is being adopted across the world. Smart homes involve introducing smartness into human dwellings to increase comfort and ease of living \cite{lashkari_energy_2019}. Another benefit is energy conservation, which is exemplified by incorporating a smart fan. Fans are used to lower temperatures in hot environments, but most often these fans have to be manually operated. This research project will provide a guide to developing an IoT level 4 autonomous smart fan that would fit nicely into a smart home \cite{rana_connected_2021}.

\subsection{Problem statement}

The world is facing an unprecedented rise in temperatures in recent years due to the depletion of the ozone layer by greenhouse gases \cite{noauthor_new_nodate}. The United Nations has listed Climate action as one of its Sustainable Development Goals in an effort to combat this situation \cite{noauthor_goal_nodate}. The high temperatures have necessitated communities to adopt climate control systems, which come down to air conditioners and fans. With the advent of smart devices and smart homes, the idea of smart air conditioners and smart fans that turn themselves on and off, and adjust their speed of rotation depending on temperature would be a welcome sight.
\par
However, most of the technological advancements have been made on air conditioners but they are very expensive to run because they use up a lot of energy. The alternative, fans, remain mostly of manual control and those that have tried to bridge the gap are semi-autonomous. The problem with these kinds of fans is that people forget to turn them off which leads to a lot of energy wastage.
\par
This, therefore, calls for the need to develop autonomous fan systems that bridges this gap. A need therefore arises to control the fan based on the temperature, position and number of occupants in a room.


\subsection{Objectives}
\subsubsection{Main objective}

To design and fabricate an \ac{IoT} Level-4 Autonomous Smart Fan.

\subsubsection{Specific objectives}

\begin{enumerate}
	\item To design and fabricate the fan blades, fan case, motor mount and stand of the smart fan. \vspace*{.2cm}
	\item To integrate a stepper motor and a brushless dc motor to the motor mount and fan blades.\vspace*{.2cm}
	\item To develop an algorithm to control the rotation of the fan blades based on the temperature, position and number of people in a room.\vspace*{.2cm}
	\item To submit a certification application at \ac{KEBS} for our product.\vspace*{.2cm}
	

\end{enumerate}


\subsection{Justification of the study}

The apparent availability of cheap and accessible technologies that are bridging the gap between manual and level-4 autonomous control lead us to conduct this study. Autonomous control is very efficient, with the optimization of energy usage contributing to the reduction in burning fossil fuels around the world. Furthermore, developing a cheap user-friendly level-4 autonomous smart fan will help grow the field of \ac{IoT}, especially in Sub Saharan Africa. This is where the world is headed, and it is only appropriate for us to try and head in the same direction. 