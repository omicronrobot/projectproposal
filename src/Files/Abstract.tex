
\section*{Abstract}
\label{sec:Abstract}

There is a rise in temperatures due to global warming and everyone is bestowed with the responsibility to reduce and possibly eradicate global warming. We too should take part in temperature regulations especially in Sub Saharan Africa so as to dampen the effect of global warming. Most common temperature regulation devices, air conditioners and fans are inefficient. Air conditioners use a lot of energy. On the other hand, most fans are manual and the smart ones are not smart enough to stop themselves in the absence of humans thereby using energy unsparingly.
\par
This project, therefore, proposes an \ac{IoT} level 4 autonomous fan system that would sense when the temperature is going up and sense the presence of people in the room and turn itself on. If the occupants leave it would automatically turn off. We intend to reduce the energy cost used in cooling houses thereby ensuring a sustainable energy consumption pattern while at large, building up sustainable cities.
\par
This system would have sensor nodes positioned at different endpoints in the room that measure the temperature of the room. The fan will only start if there are people in the room and if the temperature increases above a normal preset temperature, the fan will start running. It will adjust its speed of rotation based on the temperature difference. This will be detected by a sensor on the smart fan main unit. If people walk out the fan will sense this and stop. Generally, the fan is specified to operate between a predefined working schedule by the homeowner.
\par
This project intends to develop a smart fan product that will be applicable in society and apply for certification at the \ac{KEBS}.
