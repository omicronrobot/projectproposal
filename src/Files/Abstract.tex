\addcontentsline{toc}{section}{Abstract}
\section*{Abstract}
\label{sec:}

The majority of today's mobile platforms are non-holonomic.
They only have one or two degrees of freedom that are independent.
As a result, its maneuverability is limited, and it frequently requires a large amount of room to manage functions such as turning and parking.
By increasing a vehicle's degrees of freedom, maneuverability, it can take various complex trajectories that are difficult or impossible for non-holonomic vehicles to take.
\par
This project, therefore, proposes a prototype mobile platform with both holonomic and omnidirectional motion whilst using castor wheels.
Since castor wheels are difficult to control we intend to introduce \ac{DC} Motors that will enable the control of the castor wheels.
This allows the platform to move in complex trajectories without stopping to reset the wheels.
Ground clearance has been enhanced on the created platform, which is important for outside difficult terrain activities.
\par
Despite these advantages, controlling the mobile platform to gain the desired motion is a hard task that necessitates the use of an inventive algorithm.
To achieve three degrees of freedom motion with four independent wheels with eight degrees of freedom, there is a duplicate control problem that necessitates a complicated control system. 
We will attempt to resolve these challenges by using an forward kinematics model of the platform and a multiple input multiple output software design using advanced micro controllers.
